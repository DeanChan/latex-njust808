%  
%  \iffalse  
% 
%    The first part is a comment to the reader(s) of `floatflt.dtx'.
%
%  floatflt.dtx    Version 1.31.a, June 5, 1998 
%  (c) 1994-1998 by Mats Dahlgren  (matsd@sssk.se)
%
%  Please see the information in file `floatflt.ins' on how you 
%  may use and (re-)distribute this file.  Run LaTeX on the file 
%  `floatflt.ins' to get a .sty-file and an example.  For 
%  instructions, please run LaTeX on the file `floatflt.dtx' twice.
%
%  This file may NOT be distributed if not accompanied by 'floatflt.ins'.
%
%<*driver>
\documentclass[a4paper]{ltxdoc}
\textwidth=150mm
\textheight=210mm
\topmargin=0mm
\oddsidemargin=5mm
\evensidemargin=5mm
  \DocInput{floatflt.dtx}
%  \PrintChanges
%\end{document}
%</driver>
%  \fi
%  
%  \CheckSum{787}
%
%  \def\filename{floatflt.dtx}
%  \def\fileversion{1.31}
%  \def\filedate{1997/07/16}
%  \def\docdate{1998/06/05}
%  \date{\docdate}
% \title{Welcome to the \textsf{floatflt} package!\thanks{%
%    This document describes \textsf{floatflt} v.\fileversion{} and
%    was last updated \docdate.}}
% \author{Mats Dahlgren\footnote{Email:\ \texttt{matsd@sssk.se}\ \ \ 
% Web:\ \texttt{http://www.homenet.se/matsd/}}}  
% \begin{document}  
% \maketitle
% 
% \begin{abstract}
% The \textsf{floatflt} package is an extension
% of Thomas Kneser's style option \texttt{floatfig.sty}
% for \LaTeX{} 2.09.  The extension was done by Mats
% Dahlgren (\texttt{matsd@sssk.se} http://www.homenet.se/matsd/). 
% \\ This file and the package:
% Copyright \copyright{} 1994-1998 by Mats Dahlgren.  
% All rights reserved.
% \end{abstract}
% 
% \section{Introduction}
% The \textsf{floatflt} package defines two environments:
% \texttt{floatingfigure} and \texttt{floatingtable}.  In
% principle the two environments work the same way and do
% the same job, but there are some important differences,
% which are explained below.  The general idea is that a
% figure or table which is not very wide should be allowed
% to have regular text typsetted beside itself.  In Thomas
% Kneser's file \texttt{floatfig.sty} and in the
% documentation thereof, such figures are called
% \textit{floating figures}.  In the \textsf{floatflt}
% package, the term \textit{floating} is generally used
% for such figures and tables, despite the slight
% different use of the term \textit{float} in \LaTeX{}. 
% Hence the term \textit{floating floats} for such figures
% and tables. 
%  
%  This userguide is also available in \texttt{.pdf}-format 
%  on the internet.  It is found from my \LaTeX\ web page: 
%  \texttt{http://www.homenet.se/matsd/latex/}
% 
% \subsection{For \texttt{floatfig.sty} users}
% 
% There are five major differences in the use of
% \textsf{floatflt} as compared to \texttt{floatfig}: 
% \begin{itemize}
% \item The extension to the \texttt{floattable} environment.
% \item There is no longer any need for initialization,
%   thanks to the \verb+\AtBeginDocument+ command. 
% \item Optional arguments for the  environments which 
%   allows the user to place the floating floats left,
%   right or according to   the page number (as with
%   \texttt{floatfig.sty}); this is mainly due to the  
%   possibility to have optional arguments in newly defined
%   environments. 
% \item Optional argument for the package as a whole if
%   the user wants to   force all floating floats to be to
%   the right or left.  The placement   option at
%   \verb+\begin{floatingfigure}+ (and  
%   \verb+\begin{floatingtable}+) overrides this (and there
%   is also the   optional placement option to re-introduce
%   the page-number dependent   placement for one figure or
%   table). 
% \item Compatibility with the \textsf{multicol} 
%   package is now obtained. 
% \end{itemize}
%
% \section{Userguide}
% In the following, the usage of the package is explained
% in some detail.  Users who are unfamiliar with the 
% |floatfig.sty| package by Thomas Kneser may also find it
% useful to run \LaTeX{} on the file |floatfge.tex|.
% 
% \subsection{How to call the package}
% 
% The \textsf{floatflt} package is activated by the
% following line in your \LaTeX\ input file following your
% \verb+\documentclass+ command:\\ \hspace*{5mm}
% \verb+\usepackage[+\textit{option}\verb+]{floatflt}+\\
% The \textit{option} may be either one of the following:
% \texttt{rflt}, \texttt{lflt}, or \texttt{vflt}. 
% 
% \subsubsection{The package options}
% 
% The three package options have the following functions:
% \begin{itemize}
% \item[\texttt{rflt}]  Forces the default for the
%   floating figures and tables to be typset to the right in
%   a paragraph. 
% \item[\texttt{lflt}]  Forces the default for the
%   floating figures and tables to be typset to the left in
%   a paragraph. 
% \item[\texttt{vflt}]  Forces the  floating figures and
%   tables to be typset to the right in a paragraph on
%   odd-nubered pages, and to the left on even-numbered
%   pages; this  is also the default. 
% \end{itemize}
% The option \texttt{vflt} is actually not needed and does
% nothing but tells \LaTeX\ to use the default, it is also
% the internal default. It is supported for symmetry
% reasons with the optional arguments for the environments
% temselves. 
%  
% \subsection{How to use the \texttt{floatingfigure} environment}
%  
% In your \LaTeX\ document you invoke the
% \texttt{floatingfigure} environment by typing:\\ 
% \hspace*{5mm} 
% \verb+\begin{floatingfigure}[+\textit{option}\verb+]{+\textit{width}\verb+}+\\
% \hspace*{5mm} \textit{figure commands with or without} \verb+\caption+\\
% \hspace*{5mm} \verb+\end{floatingfigure}+\\
% The \textit{option} may be either one of the following:
% \texttt{r}, \texttt{l}, \texttt{p}, or \texttt{v}.  The
% quantity \textit{width} is the width you want your
% floating figure to have (such as ``\texttt{75mm}''). 
%   
% \subsubsection{The \texttt{floatingfigure} options}
%   
% The options to the \texttt{floatingfigure} environment
% all overrule any present package option which may be in
% effect.  The options have the following functions: 
% \begin{itemize}
% \item[\texttt{r}]  Forces the current floating figure to
%   be typset to the right in a paragraph. 
% \item[\texttt{l}]  Forces the current floating figure to
%   be typset to the left in a paragraph. 
% \item[\texttt{p}]  Forces the current floating figure to
%   be typset to the right in a paragraph if the pagenumber 
%   is odd, and to the left if even.
% \item[\texttt{v}]  Applies the package option to the
%   current figure, and if no package option is specified,
%   it forces the current floating figure to 
%   be typset to the right in a paragraph if the pagenumber 
%   is odd, and to the left if even.
% \end{itemize}
% The \texttt{p} option is used when the default
% alternating right/left typesetting of floating figures
% is desired despite the use of the \texttt{rflt} or
% \texttt{lflt} package option.  The \texttt{v} option is
% the internal default, and does not have any effect other
% than applying either the default or the option specified
% for the whole package. 
% 
% \subsection{How to use the \texttt{floatingtable} environment}
% 
% In your \LaTeX\ document you invoke the \texttt{floatingtable}
% environment by typing:\\
% \hspace*{5mm} \verb+\begin{floatingtable}[+\textit{option}\verb+]{+\\
% \hspace*{5mm} \verb+\begin{tabular}{+\textit{tabular 
%  specifiers}\verb+}+\\
% \hspace*{5mm}  \textit{here you put your table entries}\\
% \hspace*{5mm} \verb+\end{tabular}}+\\
% \hspace*{5mm} \verb+\caption+ \textit{if desired}\\
% \hspace*{5mm} \verb+\end{floatingtable}+\\
% The \textit{option} may be either one of the following:
% \texttt{r}, \texttt{l}, \texttt{p}, or \texttt{v}.  No
% specification of the width is explicitly made, instead
% \textit{the whole} \texttt{tabular} \textit{environment
% is a mandatory argument}.  Thus, it is important to
% always have the ``\verb+{+'' before
% \verb+\begin{tabular}+ and the ``extra'' ``\verb+}+''
% after \verb+\end{tabular}+.  At the present, the 
% \texttt{tabbing} environment can not be used in the
% \texttt{floatingtable} environment.  (Other commands and
% environments such as the \verb+\parbox+ command may be
% used instead of the \texttt{tabular} environment within
% a \texttt{floatingtable} environment.) 
%   
% \subsubsection{The \texttt{floatingtable} options}
%   
% The options for the \texttt{floatingtable} environment
% are exactly the same as for the \texttt{floatingfigure}
% environment, and they also have the same effects. 
%   
% \subsection{Compatibility}
%     
% The present version of \textsf{floatflt} (v.\fileversion) 
% has been tested
% with \LaTeXe{} of 1997/06/01 using MiK\TeX\ 1.07 running 
% \TeX\ 3.14159 under Win95, using the
% \textsf{article} document class.  Only changes to the
% the documentation has been made since the previous verision. 
%  
%  The
% regular \verb+\caption+ command of both the
% \verb+figure+ and \verb+tabular+ environments work fine.
%  The \verb+\hangcaption+ command of David M.~Jones
% (\texttt{dmjones@theory.lcs.mit.edu}) does not work with
% \textsf{floatflt}.  The use of \verb+\hangcaption+ gives
% the same result as \verb+\caption+.  The listing
% commands \verb+\listoffigures+ and \verb+\listoftables+
% also work properly.  However, since floating figures and
% tables normally are typsetted in a paragraph, the order
% in the lists may be strange if also regular
% \texttt{figure} and \texttt{table} environments are used
% close to the floating ones. 
%   
% Unlike \texttt{floatfig.sty}, \textsf{floatflt} works
% with the \textsf{multicol} package, provided the
% floating floats are still less wide than the columns. 
% If not, or if the floating floats and the columns have
% almost the same width, text may drift away and overlap
% text in adjacent column(s).  However, this combination 
% has several restrictions, and the placement of the
% floating floats may not always be the correct according
% to the page numbering.  Normally it also leads to many
% ''\verb+Underfull \hbox+'' warnings.  The use of this 
% combination is discouraged.
%   
% \subsection{Warnings and messages}
%   
% In Thomas Kneser's original work \texttt{floatfig.sty},
% there are several warnings issued when problems occur. 
% These are kept in \textsf{floatflt}.  Warnings will be 
% issued if two floating figures, two floating tables, 
% or if one floating table is colliding with one floating 
% figure.  Also, message(s)
% will be displayed if a floating float is moved from its
% original paragraph.  Finally, each processed floating
% float will produce a message telling which page it has
% been typsetted on. 
%   
% \section{Known problems and limitations}
%   
% The following problems and limitations are known:
% \begin{itemize}
% \item If a floating figure or table is called close to a
%   sectioning command, the floating float may be lost or
%   truncated.  This may lead to a ``collision warning'',
%   which normally has to be analyzed ``by hand''. 
% \item If two consecutive paragraphs both contain a
%   floating float, a ``collision warning'' is issued.  This
%   warning may be ignored. \item A \texttt{floatingtable}
%   environment  sometimes leads to a warning like:\\
%   \hspace*{5mm} 
% \verb+Overfull \hbox (10.78334pt too wide) in paragraph at lines 287--289+\\
%   where the line numbers refered to are those of the
%   \verb+\end{tabular}+ statement.  This could not be
%   avoided in all cases.   
% \item If you use floating floats of different kinds
%   close to each other in your input, you are likely to
%   have problems.  You may either lose one of the floating
%   floats entirely, have it overwritten by the other, or
%   have it truncated.  
% \item \LaTeX\ has problem in linebreaking short lines. 
%   This may give you bad linebreaks in the captions of
%   floating floats. 
% \item There are some problems when a floating float and a 
%   footnote appear on the same page, especially when the floating
%   float is moved to another page.  Unfortunately, this may
%   need some adjustments by hand.  (Reported by Jakob 
%   Schiotz (\texttt{schiotz@fysik.dtu.dk}).)
% \item If a floating float is placed close to a list environment 
%   (such as \texttt{enumerate} or \texttt{itemize}), the text of the 
%   list environment is likely to overwrite the floating float.  See 
%   section~\ref{LPSoln} below for a semi-automatic solution of this 
%   problem.  (The bug was reported by Rafael Gallego 
%   (\texttt{gallego@esi.us.es}).)
% \end{itemize}
% The first two of these were pointed out by Thomas Kneser
% in the original documentation of \texttt{floatfig.sty}. 
%   
% \subsection{How to Avoid the List Problem}\label{LPSoln} 
%    
% There is a semi-automatic way to avoid the ``List Problem'' 
% mentioned above.  The solution was introduced in version 1.2.  For 
% items in a list which are colliding with a floating float, there is 
% a special command \verb+\fltitem+ avaliable, which takes one mandatory
% argument. This command is to be used in place of the ordinary 
% \verb+\item+, and should have the text of the item as the argument. 
% \textit{E.~g.}\ you would write\\ \hspace*{10mm} 
% \verb+\fltitem{+\textit{text}\verb+}+ \\  instead of \\ 
% \hspace*{10mm} \verb+\item +\textit{text} \\ to avoid the text of 
% the \verb+\item+ command to be written in the floating float.  
% \verb+\fltitem+ also takes a second, optional, argument, which is an 
% extra vertical space put in after the text of the |\fltitem| 
% command.  This is meant to be used specially when there are two 
% paragraphs in the same item text, and the latter is to use the full 
% line width (\textit{i.~e.}\ when the item text ends below the 
% floating float).  So, the full syntax of the |\fltitem| command is:\\ 
% \hspace*{10mm}|\fltitem[|\textit{len}|]{|\textit{text}|}|\\ 
% which inserts the extra vertical space \textit{len} after the item  
% text.  If there is a second text which is to span the full 
% linewidth in the same item, one would write:\\
% \hspace*{10mm}|\fltitem[|\textit{len}|]{|\textit{text1}|} |\textit{text2}\\ 
% which first typesets \textit{text1} narrowed according to the 
% floating float, followed by \textit{text2} with full linewidth, and 
% an extra vertical space \textit{len} between these two texts.
% 
% There is also a second, analogous command 
% |\fltditem| which takes two mandatory and one optional argument.  
% The optional argument is the same as the optional argument of 
% |\fltitem|, the first mandatory argument is the item label, 
% \textit{e.~g.}\ the item to be described in a |description| 
% environment.  The second mandatory argument of |\fltditem| is the 
% item text.  The syntax of the |\fltditem| command is:\\ 
% \hspace*{10mm}|\fltditem[|\textit{len}|]{|\textit{label}|}{|\textit{text}|}|\\
% which produces an item with label \textit{label} and text  
% \textit{text} and extra vertical space \textit{len} after the item  
% text.  Additional text spanning the full linewidth may follow, 
% just like with |\fltitem|, but the indentation may turn out to be 
% strange.
% 
% Both |\fltitem| and |\fltditem| can be used in an ordinary 
% |itemize|, |enumerate|, or |description| environment (however, the 
% use of |\fltditem| in |enumerate| may produce strange results).  For 
% |\fltitem| and |\fltditem| to work properly, the (otherwise) 
% colliding floating float must be set flush right, that is have the 
% option |r| specified.  The use of |\fltitem| or |\fltditem| is not 
% demonstrated in the |floatexm.tex| file.  
%   
% \subsection{Sending a bug report}
%   
% Reports of new bugs in the package are most welcome. 
% However, I do not consider this to be a ``supported''
% package.  This means that there is no guarantee I (or
% anyone else) will put any effort into fixing the bug.  
% But, on the other hand, someone may try debugging, so filing a 
% bug report is always a good thing to do! (If nothing else,
% your discoveries may end up in future releases of this
% document.)  Before filing a bug report,  please take the
% following actions: 
% \begin{enumerate}
% \item Ensure your problem is not due to your inputfile;
% \item Ensure your problem is not due to your own 
%   package(s) or class(es);
% \item Ensure your problem is not covered in the section ``Known
%   problems and limitations'' above;
% \item  Try to locate the problem by writing a minimal \LaTeXe\ 
%   input file which reproduces the problem.  Include the command\\ 
%   \verb+   \setcounter{errorcontextlines}{999}+\\ in your input;
% \item Run your file through \LaTeXe;
% \item Send a description of your problem, the input file and the 
%    log file via e-mail to:\\  \texttt{matsd@sssk.se}.
% \end{enumerate}
%  
% \section{Conclusion}
%  
% Without Thomas Kneser's \texttt{floatfig.sty} as a base,
% I could never have written the \textsf{floatflt}
% package.  I hope that some users will find the package
% useful and not too bugful. :--) Comments and suggestions
% for improvements are always most welcome!\\*[3mm]
% {\itshape Enjoy your \LaTeX!\raisebox{-\baselineskip}[0pt][0pt]{mats d.}} 
%
%  \StopEventually{\par\vfill\hfill{\scriptsize Copyright
%    \copyright{} 1994-1998 by Mats Dahlgren.}} 
% \section{Package code}
%
%  This section prints the package code but without comments.
% \iffalse
%
%<*paketkod>
%  \fi
%    \begin{macrocode}
\NeedsTeXFormat{LaTeX2e}[1996/12/01]
\ProvidesPackage{floatflt}[1997/07/16 v. 1.31]
\newcounter{OptionTest}
\setcounter{OptionTest}{0}
\DeclareOption{rflt}{\setcounter{OptionTest}{1}}
\DeclareOption{lflt}{\setcounter{OptionTest}{2}}
\DeclareOption{vflt}{\setcounter{OptionTest}{0}}
\DeclareOption*{\OptionNotUsed}
\ProcessOptions
\newbox\figbox
\newbox\tabbox
\newbox\pagebox
\newcount\ffigcount
\newcount\ftabcount
\newcount\fftest
\newcount\hangcount
\newcount\nosuccesstryfig
\newcount\nosuccesstrytab
\newdimen\figgutter   \figgutter=1truepc
\newdimen\tabgutter   \tabgutter=1truepc
\newdimen\htdone      \htdone=0pt
\newdimen\pageht
\newdimen\startpageht
\newdimen\tabbredd
\newdimen\floatfltwidth
\newdimen\fltitemwidth
\newif\iftryingfig     \tryingfigfalse
\newif\iftryingtab     \tryingtabfalse
\newif\ifdoingfig      \doingfigfalse
\newif\ifdoingtab      \doingtabfalse
\newif\iffigprocessing \figprocessingfalse
\newif\iftabprocessing \tabprocessingfalse
\newif\ifpageafterfig  \pageafterfigfalse
\newif\ifpageaftertab  \pageaftertabfalse
\newif\ifoddpages
\newif\ifoutput
\newtoks\outputpretest
\newenvironment{floatingfigure}[2][v]%
{\@tfor \@tempa :=#1\do
         {\if\@tempa r\global\oddpagestrue\fi
          \if\@tempa l\global\oddpagesfalse\fi
          \if\@tempa p%
              \ifodd\c@page\global\oddpagestrue
                \else\global\oddpagesfalse\fi
          \fi
          \if\@tempa v%
             \ifnum\theOptionTest=0
                 \ifodd\c@page\global\oddpagestrue
                   \else\global\oddpagesfalse\fi
             \else
                  \ifodd\theOptionTest\global\oddpagestrue
                    \else\global\oddpagesfalse\fi
             \fi
          \fi
         }
 \expandafter\ifx\csname oldoutput\endcsname\relax% ref. TeXbook Ex.7.7
    \PackageError{floatflt}{The `floatflt' package is not initialized}
     {Try to reinstall the `floatflt' package.\MessageBreak
      Type `x' to quit or <Return> to try to go on.}\@@end\fi
 \global\everypar={\tryfig\oldeverypar}% must be set globally!
 \global\advance\ffigcount by 1
 \iffigprocessing
    {\count0=\ffigcount\advance\count0 by -1 
     \PackageWarningNoLine{floatflt}{Floating figures \the\count0\space%
       \space and \the\ffigcount\space colliding}%
    }%
 \fi 
 \iftabprocessing \PackageWarningNoLine{floatflt}{Floating figure %
   \the\ffigcount\space and floating table \the\ftabcount\space colliding}
 \fi
 \def\@captype{figure}
 \global\setlength{\floatfltwidth}{#2}
 \global\figprocessingtrue
 \global\setbox\figbox=\vbox\bgroup% begin of figbox
                               \hrule height 0pt width #2 depth 0pt%
                               \hsize=#2%
}
{
\egroup
\figinsert\par%
}
\newenvironment{floatingtable}[2][v]%
{\@tfor \@tempa :=#1\do
         {\if\@tempa r\global\oddpagestrue\fi
          \if\@tempa l\global\oddpagesfalse\fi
          \if\@tempa p%
              \ifodd\c@page\global\oddpagestrue
                \else\global\oddpagesfalse\fi
          \fi
          \if\@tempa v%
             \ifnum \theOptionTest=0
                 \ifodd\c@page\global\oddpagestrue
                   \else\global\oddpagesfalse\fi
             \else
                  \ifodd\theOptionTest\global\oddpagestrue
                    \else\global\oddpagesfalse\fi
             \fi
          \fi
         }
\expandafter\ifx\csname oldoutput\endcsname\relax% ref. TeXbook Ex.7.7
     \PackageError{floatflt}{The `floatflt' package is not initialized}
     {Try to reinstall the `floatflt' package.\MessageBreak
      Type `x' to quit or <Return> to try to go on.}\@@end\fi
\global\setbox\tabbox=\vbox\bgroup\hrule height 0pt width 0pt depth 0pt%
   \hsize=0pt\egroup  
 \global\everypar={\trytab\oldeverypar}
 \global\advance\ftabcount by 1
 \iftabprocessing
    {\count0=\ftabcount\advance\count0 by -1 
     \PackageWarningNoLine{floatflt}{Floating tables \the\count0\space%
            \space and \the\ftabcount \space colliding}%
    }
 \fi 
 \iftabprocessing \PackageWarningNoLine{floatflt}{Floating table %
   \the\ffigcount\space and floating figure \the\ftabcount\space colliding}
 \fi
 \settowidth{\tabbredd}{#2}
 \global\setlength{\floatfltwidth}{\tabbredd}
 \def\@captype{table}
 \global\tabprocessingtrue
 \global\setbox\tabbox=\vbox\bgroup% begin of tabbox
                              \hrule height 0pt width\tabbredd depth 0pt%
                               \hsize=\tabbredd
\noindent\ifnum\ftabcount >1\ifoddpages\else\hspace*{-12pt}\fi\fi%
#2\vspace{0.2\baselineskip}%
}
{
\egroup% end of \tabbox
\tabinsert\par%
}
\AtBeginDocument{%
\edef\oldoutput{\the\output}%
\output={\the\outputpretest%
           \ifoutput\oldoutput\fi}
\outputpretest={\outputtrue}
\edef\oldeverypar{\the\everypar}
}
\def\dofigtest{%
   \ifnum\outputpenalty=-10005
     \setbox\pagebox=\vbox{\unvbox255}%
     \global\pageht=\ht\pagebox
     \global\outputfalse
     \unvbox\pagebox
   \else
     \global\outputtrue
     \ifdoingfig
         \global\pageafterfigtrue
     \fi
   \fi}% 
\def\dotabtest{%
   \ifnum\outputpenalty=-10005
     \setbox\pagebox=\vbox{\unvbox255}%
     \global\pageht=\ht\pagebox
     \global\outputfalse
     \unvbox\pagebox
   \else
     \global\outputtrue
     \ifdoingtab
         \global\pageaftertabtrue
     \fi
   \fi}% 
\def\tryfig{%
      \iftryingfig
       {\everypar={\relax}\setbox0=\lastbox%
       \parindent=\wd0 \parskip=0pt \par% 
       \penalty-10005 \leavevmode}%
       \dimen0=\vsize%
       \advance\dimen0 by -\pageht%
       \advance\dimen0 by -2\baselineskip%
       \ifdim\dimen0>\ht\figbox%
         \dimen0=0.3\baselineskip
         \vrule depth \dimen0 width 0pt
         \vadjust{\kern -\dimen0%
         \vtop to \dimen0{%
             \baselineskip=\dimen0%
             \vss \vbox to 1ex{%
             \ifoddpages%
               \hbox to \hsize{\hss\copy\figbox}% 
             \else% leftsetting
               \hbox to \hsize{\copy\figbox\hss}%
             \fi% \ifodd\count0
             \vss}\null}}%
       \global\tryingfigfalse% 
       \global\doingfigtrue
       \global\startpageht=\pageht
       \global\htdone=0pt
       \dohangf
       \ifnum\nosuccesstryfig>0%
         \typeout{floatflt Message: Flt. fig. \the\ffigcount\space set on page 
             \the\count0, shifted \the\nosuccesstryfig\space par(s) forward.}%
       \else
         \typeout{Package floatflt Message: Floating figure \the\ffigcount
              \space set on page \the\count0}%
       \fi 
     \else
       \global\advance\nosuccesstryfig by 1
     \fi
     \else% 
     \ifdoingfig
       {\everypar={\relax}\setbox0=\lastbox
       \parindent=\wd0 \parskip=0pt \par
         \penalty-10005 \leavevmode}%
    \global\htdone=\pageht
    \global\advance\htdone by -\startpageht
    \ifpageafterfig
      \global\doingfigfalse
    \else
      \dimen0=\ht\figbox%
      \advance\dimen0 by 0.5\baselineskip%
      \ifdim\htdone<\dimen0%
         \dohangf
      \else
         \global\doingfigfalse
      \fi
    \fi
    \ifdoingfig\relax\else\global\figprocessingfalse\fi
  \else
    \global\outputpretest={\outputtrue}%
  \fi
\fi
}
\def\trytab{%
      \iftryingtab% 
       {\everypar={\relax}\setbox0=\lastbox%
       \parindent=\wd0 \parskip=0pt \par% 
       \penalty-10005 \leavevmode}%
       \dimen0=\vsize%
       \advance\dimen0 by -\pageht%
       \advance\dimen0 by -2\baselineskip%
       \ifdim\dimen0>\ht\tabbox%
         \dimen0=0.3\baselineskip
         \vrule depth \dimen0 width 0pt
         \vadjust{\kern -\dimen0%
         \vtop to \dimen0{%
             \baselineskip=\dimen0%
             \vss \vbox to 1ex{%
              \ifoddpages%
               \hbox to \hsize{\hss\copy\tabbox}% 
             \else% leftsetting
               \hbox to \hsize{\copy\tabbox\hss}%
             \fi% \ifodd\count0
             \vss}\null}}%
       \global\tryingtabfalse% 
       \global\doingtabtrue
       \global\startpageht=\pageht
       \global\htdone=0pt
       \dohangt
       \ifnum\nosuccesstrytab>0%
         \typeout{floatflt Message: Flt. tab. \the\ftabcount\space set on page 
             \the\count0, shifted \the\nosuccesstrytab\space par(s) forward.}%
       \else
         \typeout{Package floatflt Message: Floating table \the\ftabcount\space
                  set on page \the\count0}%
       \fi 
     \else
       \global\advance\nosuccesstrytab by 1
     \fi
     \else
     \ifdoingtab
       {\everypar={\relax}\setbox0=\lastbox
       \parindent=\wd0 \parskip=0pt \par
         \penalty-10005 \leavevmode}%
    \global\htdone=\pageht
    \global\advance\htdone by -\startpageht
    \ifpageaftertab
      \global\doingtabfalse
    \else
      \dimen0=\ht\tabbox%
      \advance\dimen0 by 0.5\baselineskip%
      \ifdim\htdone<\dimen0%
         \dohangt
      \else
         \global\doingtabfalse
      \fi
    \fi
    \ifdoingtab\relax\else\global\tabprocessingfalse\fi
  \else
    \global\outputpretest={\outputtrue}%
  \fi
\fi
}
\def\figinsert{%
   \global\nosuccesstryfig=0%
   \global\outputpretest={\dofigtest}%
   \global\tryingfigtrue \global\doingfigfalse%
   \global\pageafterfigfalse}%
\def\tabinsert{%
   \global\nosuccesstrytab=0%
   \global\outputpretest={\dotabtest}%
   \global\tryingtabtrue \global\doingtabfalse%
   \global\pageaftertabfalse}%
\def\dohangf{%
   \dimen0=\ht\figbox%
   \advance\dimen0 by -\htdone%
   \advance\dimen0 by 1.49\baselineskip%
   \hangcount=\dimen0%
   \divide\hangcount by \baselineskip%
   \dimen0=\wd\figbox%
   \advance\dimen0 by \figgutter%
   \ifoddpages%
     \global\hangafter=-\hangcount% placing right
     \global\hangindent=-\dimen0%
   \else% \ifleftsetting
     \global\hangafter=-\hangcount% placing left
     \global\hangindent=\dimen0%
   \fi
}
\def\dohangt{%
   \dimen0=\ht\tabbox%
   \advance\dimen0 by -\htdone%
   \advance\dimen0 by 1.49\baselineskip%
   \hangcount=\dimen0%
   \divide\hangcount by \baselineskip%
   \dimen0=\wd\tabbox%
   \advance\dimen0 by \tabgutter%
   \ifoddpages%
     \global\hangafter=-\hangcount% placing right
     \global\hangindent=-\dimen0%
   \else% \ifleftsetting
     \global\hangafter=-\hangcount% placing left
     \global\hangindent=\dimen0%
   \fi
}
\newcommand{\fltitem}[2][0pt]{\setlength{\fltitemwidth}{\linewidth}%
  \addtolength{\fltitemwidth}{-\floatfltwidth}%
  \addtolength{\fltitemwidth}{-0.5em}%
  \item \parbox[t]{\fltitemwidth}{#2}\\[#1]}
\newcommand{\fltditem}[3][0pt]{\setlength{\fltitemwidth}{\linewidth}%
  \addtolength{\fltitemwidth}{-\floatfltwidth}%
  \addtolength{\fltitemwidth}{-0.5em}%
  \item[#2] \parbox[t]{\fltitemwidth}{#3}\\[#1]}
%    \end{macrocode}
%  \Finale
% \iffalse
%</paketkod>
% \fi
% \end{document}
% \iffalse 
%<*exempelkod>
\documentclass[11pt]{article}
\usepackage{floatflt}
\begin{document}
\centerline{\Huge The Tale of 
\textsf{floatflt}}\bigskip
 
\noindent This is a demonstration document for the use
of the \textsf{floatflt} package.  It contains several
floating figures and tables with captions explaining how
they were called.  At the end, both a
\verb+\listoffigures+ and a \verb+\listoftables+ command
are used, resulting in the desired lists.  For more
details on how to use the \textsf{floatflt} package,
please run \LaTeX{} on the file \texttt{floatflt.dtx}.
 
The following work by Edgar Alan Poe was retrieved by
anonymous ftp from \texttt{ftp.funet.fi} in the
directory \texttt{/pub/doc/literary/etext} where it is
found in the file \texttt{telltale.poe}.  Only minor
\LaTeX\ adaptions have been done, besides the inclusion
of floating floats. 
 
\begin{center}
Internet Wiretap Edition of\\[2mm]
{\large THE TELL-TALE HEART}\\ 
by\\ {\large EDGAR ALLAN POE}\\[2mm]
From \textit{The Works of Edgar Allan Poe: Tales Vol I}\\
J. B. Lippincott Co, Copyright 1895.\\[2mm]
This text is placed into the Public Domain (May 1993).\\
\end{center}
 
\noindent\textit{\Large The Tell-Tale Heart}
 
\noindent TRUE! nervous, very, very dreadfully nervous
I had been and am; but why WILL you say
that I am mad? The disease had sharpened
my senses, not destroyed, not dulled them. Above
all was the sense of hearing acute. I heard all things
in the heaven and in the earth. I heard many things
in hell. How then am I mad? Hearken! and observe how healthily, how
calmly, I can tell you the
whole story.
 
\begin{floatingfigure}{60mm}
\begin{center}
The first figure to\\ use the environment\\ \texttt{floatingfigure}
\end{center}
\caption{The \texttt{floatingfigure} environment with \texttt{60mm}
for \textit{width} and no \textit{option}.}
\end{floatingfigure}
It is impossible to say how first the idea entered
my brain, but, once conceived, it haunted me day
and\linebreak night. Object there was none. Passion there
was none. I loved the old man. He had never
wronged me. He had never given me insult. For
his gold I had no desire. I think it was his eye!
Yes, it was this! One of his eyes resembled that
of a vulture -- a pale blue eye with a film over it.
Whenever it fell upon me my blood ran cold, and
so by degrees, very gradually, I made up my mind
to take the life of the old man, and thus rid myself
of the eye for ever.
 
\begin{floatingfigure}[r]{40mm}
\begin{center}
Another figure to\\ use the environment\\ \texttt{floatingfigure}
\end{center}
\caption{The \texttt{floatingfigure} environment with \texttt{40mm}
for \textit{width} and the \texttt{r} \textit{option}.}
\end{floatingfigure}
Now this is the point. You fancy me mad. Madmen 
know nothing. But you should have seen me.
You should have seen how wisely I proceeded --
with what caution -- with what foresight, with what
dissimulation, I went to work! I was never kinder
to the old man than during the whole week before
I killed him. And every night about midnight I
turned the latch of his door and opened it oh, so
gently! And then, when I had made an opening
sufficient for my head, I put in a dark lantern all
closed, closed so that no light shone out, and then
I thrust in my head. Oh, you would have laughed
to see how cunningly I thrust it in! I moved it
slowly, very, very slowly, so that I might not 
disturb the old man's sleep. It took me an hour to
place my whole head within the opening so far that
I could see him as he lay upon his bed. Ha! would
a madman have been so wise as this? And then
when my head was well in the room I undid the
lantern cautiously -- oh, so cautiously -- cautiously
(for the hinges creaked), I undid it just so much
that a single thin ray fell upon the vulture eye.
And this I did for seven long nights, every night
just at midnight, but I found the eye always closed,
and so it was impossible to do the work, for it was
not the old man who vexed me but his Evil Eye.
And every morning, when the day broke, I went
boldly into the chamber and spoke courageously to
him, calling him by name in a hearty tone, and 
inquiring how he had passed the night. So you see
he would have been a very profound old man, indeed, 
to suspect that every night, just at twelve, I
looked in upon him while he slept.
 
\begin{floatingtable}{
\begin{tabular}{cccc}
$x$ & $x^2$ & $x^3$ & $x^4$ \\ \hline
1 & 1 & 1 & 1 \\
2 & 4 & 8 & 16 \\
3 & 9 & 27 & 81 \\
\end{tabular}}
\caption{The \texttt{floatingtable} environment with no \textit{option}.}
\end{floatingtable}
Upon the eighth night I was more than usually
cautious in opening the door. A watch's minute
hand moves more quickly than did mine. Never
before that night had I felt the extent of my own
powers, of my sagacity. I could scarcely contain
my feelings of triumph. To think that there I was
opening the door little by little, and he not even to
dream of my secret deeds or thoughts. I fairly
chuckled at the idea, and perhaps he heard me, for
he moved on the bed suddenly as if startled. Now
you may think that I drew back -- but no. His room
was as black as pitch with the thick darkness (for
the shutters were close fastened through fear of
robbers), and so I knew that he could not see the
opening of the door, and I kept pushing it on
steadily, steadily.
 
\begin{floatingtable}[l]{
\begin{tabular}{ccc}
$\alpha$ & $\sin\alpha$ & $\cos\alpha$ \\ \hline
0 & 0 & 1 \\
$\pi$ & 0 & $-1$ \\
$2\pi$ & 0 & 1 \\
\end{tabular}}
\caption{The \texttt{floatingtable} environment with the \texttt{l} 
\textit{option}.}
\end{floatingtable}
I had my head in, and was about to open the
lantern, when my thumb slipped upon the tin fasten\-ing, 
and the old man sprang up in the bed, crying
out, ''Who's there?''
 
I kept quite still and said nothing. For a whole
hour I did not move a muscle, and in the meantime
I did not hear him lie down. He was still sitting
up in the bed, listening; just as I have done night
after night hearkening to the death watches in the
wall.
 
\begin{floatingfigure}[l]{50mm}
\begin{center}
\Large A Figure!
\end{center}
\caption{The \texttt{floatingfigure} environment 
with \textit{width} set to \texttt{50mm}  and the 
\texttt{l} \textit{option}.}
\end{floatingfigure}
Presently, I heard a slight groan, and I knew it
was the groan of mortal terror. It was not a groan of
pain or of grief -- oh, no! It was the low stifled sound
that arises from the bottom of the soul when 
overcharged with awe. I knew the sound well. Many
a night, just at midnight, when all the world slept,
it has welled up from my own bosom, deepening,
with its dreadful echo, the terrors that distracted
me. I say I knew it well. I knew what the old
man felt, and pitied him although I chuckled at
heart. I knew that he had been lying awake ever
since the first slight noise when he had turned in
the bed. His fears had been ever since growing
upon him. He had been trying to fancy them
causeless, but could not. He had been saying to
himself, ''It is nothing but the wind in the chimney,
it is only a mouse crossing the floor,'' or, ''It is merely
a cricket which has made a single chirp.'' Yes he
has been trying to comfort himself with these 
suppositions; but he had found all in vain. ALL IN VAIN,
because Death in approaching him had stalked with
his black shadow before him and enveloped the
victim. And it was the mournful influence of the
unperceived shadow that caused him to feel, although 
he neither saw nor heard, to feel the presence
of my head within the room.
 
When I had waited a long time very patiently
without hearing him lie down, I resolved to open
a little -- a very, very little crevice in the lantern.
So I opened it -- you cannot imagine how stealthily,
stealthily -- until at length a single dim ray like the
thread of the spider shot out from the crevice and
fell upon the vulture eye.
 
It was open, wide, wide open, and I grew furious
as I gazed upon it. I saw it with perfect distinctness 
-- all a dull blue with a hideous veil over it that
chilled the very marrow in my bones, but I could
see nothing else of the old man's face or person, for
I had directed the ray as if by instinct precisely upon
the damned spot.
 
And now have I not told you that what you mis-
take for madness is but over-acuteness of the senses?
now, I say, there came to my ears a low, dull, quick
sound, such as a watch makes when enveloped in
cotton. I knew that sound well too. It was the
beating of the old man's heart. It increased my fury
as the beating of a drum stimulates the soldier into
courage.
 
\begin{floatingfigure}[p]{90mm}
\begin{center}
A rather wide figure which still uses the \\ 
\texttt{floatingfigure} environment.
\end{center}
\caption{A \texttt{floatingfigure} environment 
which uses \texttt{90mm} for \textit{width} and 
the \texttt{p} \textit{option}.}
\end{floatingfigure}
But even yet I refrained and kept still. I scarcely\linebreak 
brea\-thed. I held\linebreak the lantern motionless. I tried
how\linebreak stead\-ily I could\linebreak 
maintain the ray upon the eye.
Meantime the hellish tattoo of the heart increased.
It grew quicker and quicker, and louder and louder,
every instant. The old man's terror must have
been extreme! It grew louder, I say, louder every
moment! -- do you mark me well? I have told you
that I am nervous: so I am. And now at the dead
hour of the night, amid the dreadful silence of that
old house, so strange a noise as this excited me to
uncontrollable terror. Yet, for some minutes longer
I refrained and stood still. But the beating grew
louder, louder! I thought the heart must burst.
And now a new anxiety seized me -- the sound would
be heard by a neighbour! The old man's hour had
come! With a loud yell, I threw open the lantern
and leaped into the room. He shrieked once -- once
only. In an instant I dragged him to the floor, and
pulled the heavy bed over him. I then smiled
gaily, to find the deed so far done. But for many
minutes the heart beat on with a muffled sound.
This, however, did not vex me; it would not be
heard through the wall. At length it ceased. The
old man was dead. I removed the bed and examined
the corpse. Yes, he was stone, stone dead. I placed
my hand upon the heart and held it there many
minutes. There was no pulsation. He was stone
dead. His eye would trouble me no more.
 
\begin{floatingtable}[r]{
\begin{tabular}{l}
Two lines in one table with one single \\
column is enough!
\end{tabular}}
\caption{This \texttt{floatingtable}  uses the \texttt{r} 
\textit{option}.}
\end{floatingtable}
If still you think me mad, you will think so no
longer when I describe the wise precautions I took
for the concealment of the body. The night waned,
and I worked hastily, but in silence.
 
I took up three planks from the flooring of the
chamber, and deposited all between the scantlings.
I then replaced the boards so cleverly so cunningly,
that no human eye -- not even his -- could have 
detected anything wrong. There was nothing to wash
out -- no stain of any kind -- no blood-spot whatever.
I had been too wary for that.
 
When I had made an end of these labours, it was
four o'clock -- still dark as midnight. As the bell
sounded the hour, there came a knocking at the
street door. I went down to open it with a light
heart, -- for what had I now to fear? There entered
three men, who introduced themselves, with perfect
suavity, as officers of the police. A shriek had been
heard by a neighbour during the night; suspicion
of foul play had been aroused; information had been
lodged at the police office, and they (the officers)
had been deputed to search the premises.
 
\begin{floatingtable}[p]{
\begin{tabular}{|l|l|l|} \hline
English Word & Swedish Word & Dutch Word \\ \hline
read & l\"asa & lesen \\
speak & tala & spreken\\
write & skriva & schrijven \\ \hline
\end{tabular}}
\caption{A \texttt{floatingtable}  with the \texttt{p} 
\textit{option}.}
\end{floatingtable}
I smiled, -- for what had I to fear? I bade the
gentlemen welcome. The shriek, I said, was my
own in a dream. The old man, I mentioned,\linebreak was
absent in the country. I took my visitors all over
the house. I bade them search -- search well. I led
them, at length, to his chamber. I showed them his
treasures, secure, undisturbed. In the enthusiasm
of my confidence, I brought chairs into the room,
and desired them here to rest from their fatigues,
while I myself, in the wild audacity of my perfect
triumph, placed my own seat upon the very spot
beneath which reposed the corpse of the victim.
 
The officers were satisfied. My MANNER had 
convinced them. I was singularly at ease. They sat
and while I answered cheerily, they chatted of
familiar things. But, ere long, I felt myself getting
pale and wished them gone. My head ached, and
I fancied a ringing in my ears; but still they sat,
and still chatted. The ringing became more distinct: 
I talked more freely to get rid of the feeling:
but it continued and gained definitiveness -- until,
at length, I found that the noise was NOT within my
ears.
 
No doubt I now grew VERY pale; but I talked
more fluently, and with a heightened voice. Yet
the sound increased -- and what could I do? It was
A LOW, DULL, QUICK SOUND -- MUCH SUCH A SOUND AS A
WATCH MAKES WHEN ENVELOPED IN COTTON. I gasped for
breath, and yet the officers heard it not. I talked
more quickly, more vehemently but the noise
steadily increased. I arose and argued about trifles,
in a high key and with violent gesticulations; but
the noise steadily increased. Why WOULD  they not
be gone? I paced the floor to and fro with heavy
strides, as if excited to fury by the observations of
the men, but the noise steadily increased. O God!
what COULD I do? I foamed -- I raved -- I swore! I
swung the chair upon which I had been sitting, and
grated it upon the boards, but the noise arose over
all and continually increased. It grew louder --
louder -- louder! And still the men chatted 
pleasantly, and smiled. Was it possible they heard
not? Almighty God! -- no, no? They heard! --
they suspected! -- they KNEW! -- they were making
a mockery of my horror! -- this I thought, and this
I think. But anything was better than this agony!
Anything was more tolerable than this derision! I
could bear those hypocritical smiles no longer! I 
felt that I must scream or die! -- and now -- again
-- hark! louder! louder! louder! LOUDER! --
 
''Villains!'' I shrieked, ''dissemble no more! I
admit the deed! -- tear up the planks! -- here, here!
-- it is the beating of his hideous heart!''
 
\rightline{\ensuremath{\mathcal{END}}.}
 
\listoffigures
\listoftables
 
%</exempelkod>
%<exempelkod>\end{document}
%<*gammalkod>
 \documentclass[11pt]{article}
 %
 \textheight=1.1\textheight
 \textwidth=1.15\textwidth
 \oddsidemargin=0pt
 \evensidemargin=0pt
 %
 \begin{document}
 %
 \title{\LaTeX-Paragraphs Floating around Figures}
 \author{{\it Thomas Kneser}\\{\footnotesize
 Gesellschaft f\"ur wissenschaftliche Datenverarbeitung 
  mbH G\"ottingen, FRG}}
 \date{August 20, 1990}
 \maketitle
 %
 \begin{abstract}
 %
 \noindent Frequently figures do not fill the full pagewidth. 
 If the width of such figures is only half of the 
 pagewidth or even less,
 textlines should be set beside the figures,
 or -- from another point of view -- figures should `float'
 in paragraphs.
 This article presents the \LaTeX{} style option {\tt FLOATFIG} 
 which can be used to set such Floating Figures as easily 
 as \LaTeX's standard figures.
 \end{abstract}
 % 
 \section{The name of the game}
 The macros which make up the {\tt FLOATFIG} style option
 are based on PLAIN-\TeX\ macros developed by Thomas Reid
 (TUGBoat Vol. 8 \# 3 page 315), who pointed out
 how to set figures
 {\bf right justified} with paragraphs floating around them.
 For such objects
 Th. Reid has chosen the term `Floating Figures'.
  
 This choice might cause some confusion,
 when we adapt these macros for \LaTeX, 
 since Leslie Lamport uses the term `float' for objects
 which are realized as \TeX-\verb+\inserts+.
 While L. Lamports floats are floating in `main vertical list',
 floats introduced by Th. Reid
 are floating in paragraphs.
 For the following we adopt the latter meaning.  
 \section{How to use it}
 The Floating Figures style option is fully compatible with 
 \LaTeX's standard figure facility:
 \begin{enumerate}
 \item Floating Figures and standard figures may be requested
       in any sequence,
 \item Floating Figures can be captioned like standard figures,
 \item captioned Floating Figures are inserted in the list of figures 
       which may be printed by the standard 
       \verb+\listoffigures+ command.
 \end{enumerate}
 A Floating Figure may be requested as follows:
 \begin{verbatim}
   \documentstyle[floatfig]{article}
   \begin{document}
   \initfloatingfigs
           .
           . 
   \begin{floatingfigure}{5.6cm}
   \vspace{6.cm}
   \caption{Intermolecular potential K-Xe}% optional !
   \end{floatingfigure}
           .
           .
   \end{document}
 \end{verbatim}
 where {\tt 5.6cm} specifies the width of the figure space. 
 A Floating Figure may only be requested in vertical mode, that is
 between paragraphs.
  
 A Floating Figure will be set as soon as possible after its request
 has be encountered by \TeX. 
 That means,
 it will be tested if there is enough vertical space 
 on the current page;
 otherwise the figure moves to the next page. 
  
 Floating Figures are set {\bf alternating}, 
 that is on the right hand side on odd and
 on the left hand side on even numbered pages.
 %
 \subsection{Restrictions}
 \begin{enumerate}
 \item The {\tt FLOATFIG} style option may not be combined with the
 {\tt TWOCOLUMN} style option,
 \item a Floating Figure will never appear in a paragraph 
 which begins on top of a page.
 \end{enumerate}
 %
 \section{About the internals}
 %
 We have extended the macros designed by Th.~Reid with regard to:
 \begin{enumerate}
 \item implantation into the \LaTeX{} context,
 \item alternating setting of Floating Figures as explained above,
 \item generation of warning messages for `collisions' of 
       two Floating Figures. 
 \end{enumerate}
 %
 \subsection{Implantation into the \LaTeX{} context}
 The PLAIN-\TeX{} implementation by Th.~Reid
 is based on a redefined \verb+\output+ routine:
 %
 \begin{verbatim}
   \edef\oldoutput{\the\output}%
   \output={\the\outputpretest\ifoutput\oldoutput\fi}
   \outputpretest={\outputtrue}
 \end{verbatim}
 %
 If a Floating Figure is requested, 
 the content of the
 \verb+\outputpretest+ token register is prepared to decide:
 \begin{enumerate}
 \item if there is enough vertical space to set the Floating Figure,
 \item if setting of a Floating Figure is in progress or
 \item if indeed the current page has to be sent to the DVI file.
 \end{enumerate}
 %
 In general, \TeX{} has to deal with more than one paragraph until
 a Floating Figure will be completely processed.
 During this process
 the redefined \verb+\output+ routine is
 called at the begin of {\bf every} paragraph; 
 this is done indirectly
 by expanding the control sequence \verb+\tryfig+.
 Therefore, the \verb+\everypar+ token
 list is prepared by the following command sequence:
 \begin{verbatim}
   \edef\oldeverypar{\the\everypar}
   \everypar={\tryfig\oldeverypar}
 \end{verbatim}
 Now \verb+\tryfig+ triggers the (modified)
 \verb+\output+ routine, which then does the decisions
 and actions mentioned above.
  
 Adopting this concept when using the macros in the \LaTeX{} context,
 we are faced with the following problems:
 \begin{enumerate}
 \item At the time \hbox{\tt FLOATFIG.STY} is read in, 
       the \verb+\output+ routine is still undefined;
       its definition is retarded until
       \verb+\begin{document}+ will be expanded;
       so the redefinition of the \verb+\output+ routine has to be
       done after \verb+\begin{document}+
       by the command \verb+\initfloatingfigs+
       (see section `Known Problems' below).
 \item There are situations where
       \LaTeX{} decides to redefine the \verb+\everypar+ token list
       without saving of the former content;
       this occurs for instance when expanding 
       a \verb+\section+ control sequence.
       We overcome this by redefining \verb+\everypar+ whenever
       the \verb+\floatingfigure+ environment is entered.
 %
       Hence to avoid problems, a Floating Figure should be requested
       early enough before any sectioning control sequence
       (see also subsection `Misleading collision warnings').\hfil\break
 %
       Furthermore, the concurrent definitions of
       \verb+\everypar+ are the reason why
       Floating Figures cannot move across section boundaries.
 \end{enumerate}
 %
 \subsection{Alternating figure setting}
 The problem to be solved
 is to decide if a certain figure has to be set 
 left- or rightjustified. 
 This decision has to made according to
 the value of the pagecount (left if even, right if odd).
 That is we are dealing with the wellknown problem 
 to associate a certain part of input text with the number
 of the page on which it will be set finally.
  
 As pointed out by D.E. Knuth in `The {\TeX}book' this association
 is done not before \verb+\output+ routine time.
 But the problem is not so hard to solve, 
 since in Th.~Reid's version there is already a modified 
 \verb+\output+ routine which decides if a certain figure will
 fit on the current page.
 As a by-product of this decision one easily gains
 the information `odd' or `even' for the pagecount
 of the current page.
 So our problem reduces to the following simple decision:
 \begin{verbatim}
   \ifodd\count0 %
      \hbox to \hsize{\hss\copy\figbox}% 
      \global\oddpagestrue
    \else% leftsetting
      \hbox to \hsize{\copy\figbox\hss}%
      \global\oddpagesfalse
    \fi% \ifodd\count0
 \end{verbatim}
 %
 \subsection{Collisions of Floating Figures}
 We define a collision as a situation where:
 \begin{enumerate}
 \item a Floating Figure is requested before the predecessor
       has been finished,
 \item some sectioning is requested before a Floating Figure has been
       finished.
 \end{enumerate}
  
 While the {\tt FLOATFIG} style option cannot avoid such collisions,
 it will recognize them.
 For diagnostic purposes we have therefore defined 
 the switch \verb+\iffigprocessing+ and another count register called
 \verb+\ffigcount+.
 This count register is used to attach 
 a sequence number to each Floating Figure,
 so they can be identified uniquely within collision warning messages. 
 These sequence numbers are not to be confused 
 with the figure count maintained by standard \LaTeX.
 %
 \section{Known Problems}
 \subsection{Need of Initialization}
 The present version of the style option needs to be initialized
 by control sequence 
 {\tt\verb+\initfloatingfigs+} as mentioned above.\footnote{%
 This is not needed with \LaTeXe{}. /MD}
  
 We hope a later version will initialize itself, 
 when the first request to a Floating Figure is encountered. 
 One problem with such an automatic initialization seems 
 to be, 
 that we are grouped down
 since we are inside a \LaTeX{} environment.
 Our first attempt failed 
 to make the respective settings \verb+\global+.
 %
 \subsection{Misleading collision warnings}
 As mentioned above, Floating Figures do not move 
 across section boundaries.
 If a Floating Figure is requested near a section end, 
 the figure will be truncated, 
 if it does not fit in the current section.
  
 If this occurs for instance with floating figure number~4,  
 a collision will be reported when the request for
 floating figure number~5 is encountered;
 so the respective warning message may be substantially retarded.
 In fact, the message will tell us that there is a problem with figure~4, 
 but there will be no further hint.
 So one has to analyze `by hand' that the problem is {\bf not} caused
 by collision with figure~5 but with a section heading;
 even worse: if there is no floating figure~5, one gets no warning message
 at all. 
  
 Furthermore, warning messages will be generated if a Floating Figure
 ends in paragraph $n$ and the next one begins with paragraph $n+1$. 
 These warning messages are to be ignored; they are due to some retardation
 caused by the \verb+\everypar+ mechanism.
 %
 \section{Conclusions}
 Working on \texttt{FLOATFIG.STY} we had some unexpected problems 
 which were caused
 by \LaTeX's somewhat unsafe assignments to the \verb+\everypar+ token list.
 This is due to the fact that the use
 of this token list is fundamental
 to the algorithm designed by Th.~Reid.
 On the other hand we did expect problems to couple
 the Floating Figures with \LaTeX's figure caption adminstration, 
 that is to achieve a single figure caption numbering for
 the standard figures and the Floating Figures and to get both types
 listed by the \verb+\listoffigures+ control sequence.
 %
 But these problems could easily be solved
 by the following local definiton within the {\tt FLOATFIG} environment:
 \begin{verbatim}
   \def\@captype{figure}
 \end{verbatim}
 Obviously this is due to the fact 
 that \LaTeX's caption apparatus is thoroughly parameterized.
  
 The \verb+FLOATFIG.STY+ file is stored in the 
 {\tt EARN/BITNET} listserver
 in Heidelberg; \hfil\break
 {\tt VM/CMS} users should enter the commands:
 \begin{verbatim}
   TELL LISTSERV AT DHDURZ1 GET LATEXSTY INDEX
   TELL LISTSERV AT DHDURZ1 GET LATEXSTY README
   TELL LISTSERV AT DHDURZ1 GET LATEXSTY FILELIST
   TELL LISTSERV AT DHDURZ1 GET FLOATFIG ZOOUUE LATEXSTY
 \end{verbatim}
 to obtain information about how to obtain the style option file.
\subsection{Note by Mats Dahlgren}
%
Several features described in this document still apply to the
\LaTeXe\ \texttt{floatflt} package, whereas others do not.  Please
run \LaTeX{} on \texttt{floatflt.dtx} twice and read the resulting 
documentation before you use the package.
(This note was added on October 20, 1994; changed on December 27, 
1994.)

  
 %\tableofcontents
%</gammalkod>
%<gammalkod> \end{document}
% \fi
%
%  \Finale
%
\endinput 

